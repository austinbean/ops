\documentclass[10pt]{article}

\usepackage{amsmath}
\usepackage{amsfonts}
\usepackage{amssymb}
\usepackage{hyperref}
\usepackage{graphicx}
\usepackage{tikz}
\usetikzlibrary{arrows, chains}

\usepackage{bm}

\begin{document}

\title{Notes: Treating Pain After Opioids}
\maketitle
\section{Todo}

\begin{itemize}
\item ICD codes
\item Merging to other data sources? Costs?
\item Pull out data
\item For data pull, we need to know what to pull
\item Make a list of potential identification strategies
\item Make a list of questions and target journals
\end{itemize}

\section{Introduction}
Much attention has been paid to inappropriate opioid prescribing and its role in the opioid epidemic, and recent studies suggest that the various interventions and increased awareness resulting from the opioid epidemic have resulted in restricted access to prescription opioids. As a result, patients who have relied on prescription opioids, particularly those patients with chronic pain conditions, have reported substantial disruptions in their treatment recently. 

Yet little attention has been paid to these patients and what happens to them after the imposition of opioid supply restrictions. This project will examine the treatment and diagnostic trajectory of patients who had been using prescription opioids after opioid supply restrictions are imposed. Treatments will include other medications, like NSAIDs and psychotropic medications, as well as non-medication alternatives such as physical therapy.

Despite the decline in opioid prescriptions, there is little research on the treatment trajectories that patients typically take after they stop using opioids. The two classes of drugs that are most likely to be substituted for opioids are probably those referred to in most clinical guidelines: NSAIDs and selected psychotropic medications (CDC 2018). Notably, the evidence of their efficacy in the treatment of conditions like chronic lower back pain is as weak as opioids. Moreover, some of these medications have potentially dangerous side effects. For example, diclofenac, which is an NSAID commonly used for arthritis, has been linked to an increased chance of adverse cardiovascular and gastrointestinal events, including heart attacks and perforations of the stomach and intestines.

\subsection{Research questions}
Add here: concrete research questions with potential journal outlets
\section{Identification strategies or potential policies}
Supply side policies to curb prescription opioid abuse:
\begin{itemize}
\item Prescription Drug Monitoring Programs (PDMPs)
\item Medicaid Lock-In Programs
\item Pain clinic laws
\item Enhanced diversion control
\item Black box warnings
\item Abuse-deterrent drug formulations: introduction of OxyContin in 2010 (Alpert et al). 
\end{itemize}
The emphasis on supply-side interventions is consistent with the National Drug Control Strategy of the United States more broadly. 
Less attention and funding have been directed to demand-side interventions, such as prevention and substance abuse treatment.




\section{Data}

\subsection{Sample}
In this project, our primary focus will be patients in chronic pain who discontinue prescription opioids (although we intend to examine post-opioid treatment trajectories for some acute pain conditions for comparison’s sake). These patients are more likely to use opioids long-term, and therefore are more likely to be more impacted by a restriction of supply of prescription opioids. We anticipate a large sample, as millions of Americans use opioids to manage pain for extended periods of time—in 2013 and 2014, 80\% of prescriptions for opioids lasted longer than 90 days compared to 45\% in 1999 and 2000 (Mojtabai 2018). 

Excluding cancer, the most common treatment indications for long-term opioid use was back and joint pain (Shah et al. 2017). Therefore, we will start by focusing on patients with chronic lower back pain and osteoarthritis.

\subsection{Variables}

\begin{itemize}
\item Opioid substitutes: NSAIDs and selected psychotropic medications (CDC 2018) (DDD \& number of prescriptions)
\item 
\end{itemize}



No you are not an idiot.  
COntrol your anxiety with drugs.  

Now I'm going to change something again.  No more drugs.

\end{document}
